\documentclass{article}

%Paquetes
\PassOptionsToPackage{table,dvipsnames,x11names}{xcolor}
\usepackage[spanish]{babel}
\usepackage[hidelinks]{hyperref}
\usepackage{listings}
\usepackage{fancyhdr}
\usepackage{amssymb}
\usepackage{textcomp} 
\usepackage{geometry}
\usepackage{enumitem}
\usepackage{amsmath}
\usepackage{twemojis}
\usepackage{logix}
\usepackage{xcolor}
\usepackage{colortbl}
\usepackage{tikz}
\usepackage{booktabs}
\usepackage{float}
\usetikzlibrary{calc,positioning,matrix}
\pgfdeclarelayer{background}
\pgfsetlayers{background,main}

\lstdefinestyle{mystyle}{
  backgroundcolor=\color{backcolour},   
  commentstyle=\color{codegreen},
  keywordstyle=\color{magenta},
  numberstyle=\tiny\color{codegray},
  stringstyle=\color{codepurple},
  basicstyle=\ttfamily\footnotesize,
  breakatwhitespace=false,         
  breaklines=true,                 
  captionpos=b,                    
  keepspaces=true,                 
  numbers=left,                    
  numbersep=5pt,                  
  showspaces=false,                
  showstringspaces=false,
  showtabs=false,                  
  tabsize=2
}

\lstset{style=mystyle}

%Documento
\begin{document}
  \begin{titlepage}
  	\vspace*{0.8cm}
  	\centering
  	{\bfseries\huge Universidad Nacional Autónoma de México \par}
  	\vspace{1cm}
  	{\scshape\huge Facultad de Ciencias \par}
  	\vspace{2cm}
  	{\scshape\Huge Práctica 06 \par}
  	\vfill
		{\scshape\Large Coprocesador y punto flotante \par}
  	\vspace{1.5cm}
  	{\scshape\Large Organización y Arquitectura de Computadoras \par}
  	\vfill
  	{\scshape\Large Semestre: 2026-1\par}
  	\vspace{1.5cm}
		{\Large Fecha de entrega: 30 de Octubre de 2025\par}
		\vspace{1.5cm}
		{\Large Integrantes del equipo:\par}
		\begin {itemize}
		\item Monroy Flores Alexa Sofía
		\item Salazar Enríquez Luis Alberto
		\end{itemize}
  \end{titlepage}

	\newgeometry{left=1.8cm, right=2cm, top=2.8cm, bottom=2.4cm}
	\pagestyle{fancy}
	\fancyhead[RO,L]{\textbf{\makebox[\textwidth][r]{\ensuremath{\OutlineCurvedDiamond}Práctica 6\ensuremath{\LoopArrowRight}Monroy Flores\ensuremath{\parallel}Salazar Enriquez}}}
	\section*{Preguntas}
	\begin{enumerate}
    \item \textbf{¿Porque las operaciones de flotantes no tienen número inmediato?}\\
		Por que al ser números representados en punto flotante, pierden precisión en los dígitos que los componen debido a que su representación en binario puede
		no ser exacta, haciendo que las operaciones entre ellos no tengan un número inmediato.
    \item \textbf{Cuando tenemos un numero doble guardado a lo largo de 2 registros, ¿Que datos guarda cada registro?}\\
		En el primer registro se guardan los primeros 32 bits del double que son los más representativos conteniendo el signo, exponente y la parte alta de la mantisa, 
		mientras que en el segundo registro se guardan los 32 bits restantes del número, conteniendo el resto o parte baja de la mantisa.
    \item \textbf{En MARS, en la barra de herramientas, en la pestaña de Tools, existe la herramienta llamada MIPS X-Ray, conecta esta herramienta y corre un programa linea por linea. ¿Que significan los números resaltados de color magenta, verde, azul y azul claro que se encuentran abajo de la instrucción?}
    \begin{itemize}
      \item Magenta: es "opcode" que es el codigo de operación que se va a realizar.
      \item Verde: es "rs" es el registro destino.
      \item Azul: es "rt" que es es registro destino.
      \item Azul claro: es immediate o valor inmediato.
    \end{itemize}
  \end{enumerate}
  \section*{Notas de ejercicios}
    \begin{itemize}
      \item \textbf{Ejercicio 1}
	  	\begin{figure} [H]
			\centering \includegraphics[width=0.6\textwidth]{leibniz.png}
    		\caption{Fragmento de código de Java de la serie de Leibniz}
		\end{figure}
      \item \textbf{Ejercicio 4}\\
        ¿El resultado es el mismo? No, esto debido a que en preisición doble o double si da un resultado bien, mientras que en preisción simple o float no da un resultado concreto, da como resultado "Infinity", esto a causa que el resultado es demasiado grande que llega a un desbordamiento de memoria.
    \end{itemize}
	\restoregeometry
\end{document}
